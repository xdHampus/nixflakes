\section{Analysis}
In this section the problem domain is analyzed, to help in constructing requirements for the project. 

\subsection{Subject relevant to the problem domain}
Some text about the subject relevant to the problem domain.

\subsection{Use cases}
Some use cases will be described in this section. The use cases are:
\begin{enumerate}
    \item Creating a new documentation
    \item Using Nix to manage the building of the documentation
\end{enumerate}

Detailed use cases can then be described in the following way.
This is done to make it easier to understand the use cases and avoid room for misinterpretation.


\begin{table}[H]
\caption{Use case 1: Creating a new documentation}
\centering
\begin{tabular}{|l|l|}
\hline
Use Case:        & \textbf{Creating a new documentation}                                                           \\ \hline
Id:              & \textbf{UC1}                                                                                    \\ \hline
Primary actor:   & \textbf{Writer}                                                                                 \\ \hline
Secondary actor: & \textbf{None}                                                                                   \\ \hline
Steps:           & \begin{tabular}[c]{@{}l@{}} 
    \tabitem Copy project\\  
    \tabitem Modify project\\ 
    \tabitem Compile and publish documentation  
                   \end{tabular}                                                                                   \\ \hline
Pre condition    & N/A                                                                                             \\ \hline
Post condition   & N/A                                                                                             \\ \hline
Notes:           & The project is used and adapted as needed by the writer                                         \\ \hline
\end{tabular}
\label{tab:uc1}
\end{table}



